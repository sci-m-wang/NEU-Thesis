\chapter{Introduction}\label{chap:introduction}

\section{Background}

Considering that many students may lack experience in using \LaTeX{}, NEU-Thesis has highly encapsulated the complexity of \LaTeX{}, opening up a simple interface for easy use. At the same time, some of the main difficulties in writing a dissertation with \LaTeX{}, such as graphing, tabulation and bibliographic indexing, are explained in detail, and corresponding code samples are provided. After understanding the above issues, there will be no substantial difficulties for beginners in writing a dissertation using this template. So, if you are a beginner, please don't just give up, because I, also a beginner, understand very well the importance of making \LaTeX{} easy to use, and that is what NEU-thesis seeks and embodies.

This template is based on the Chinese Academy of Sciences University dissertation template causthesis template. NEU-thesis template meets the latest Northeastern University doctoral dissertation layout requirements and cover print settings. The template is compatible with the following operating systems: Windows, Linux, MacOS, and \LaTeX{}compilation engines: pdflatex, xelatex, lualatex. features such as Chinese bookmarks, Chinese rendering, Chinese bold display, copying text from PDF to other text editors are supported. In addition, the document structure of the templates has been carefully designed and compilation scripts have been written to improve the ease of use and efficiency of the templates.

The goal of NEU-thesis is to simplify the writing of a dissertation by taking advantage of the \LaTeX{} feature of separating the formatting from the content so that once the template has been formatted, the author can focus only on the content of the dissertation. At the same time, NEU-thesis has a clean and consistent code structure and concise notes, and a careful reading of the documentation provides a window for beginners to learn \LaTeX{}.

\section{System requirements}\label{sec:system}

\href{https://github.com/sci-m-wang/NEU-Thesis-en}{\texttt{NEU-thesis}} package can be used in the current mainstream  \href{https://en.wikibooks.org/wiki/LaTeX/Introduction}{\LaTeX{}} compilation system, such as C\TeX{} (Please do not confuse the C\TeX{} Suite, a \LaTeX{} compilation system that integrates many \LaTeX{} components, with the ctex macro package, which has been discontinued and \textbf{is no longer recommended}. The \href{https://ctan.org/pkg/ctex?lang=en}{ctex} macro package, like NEU-thesis, is the \LaTeX{} command set with an active maintenance status and is integrated by default by mainstream \LaTeX{} compilation systems, and is the core architecture of almost all \LaTeX{} Chinese documentation.) , MiK\TeX{} (less stable maintenance, \textbf{less recommended}), \TeX{}Live. recommended \href{https://en.wikibooks.org/wiki/LaTeX/Installation}{\LaTeX{} compilation system} and \href{https ://en.wikibooks.org/wiki/LaTeX/Installation}{\LaTeX{}Text Editor} for

\LaTeX{} compilation systems (e.g. \TeX{}Live) are used to provide the compilation environment, \LaTeX{} text editors (e.g. Texmaker) are used to edit \TeX{} source files. Please download the installer from the official website of each software, do not use other program sources. After successful installation of \textbf{\LaTeX{} compilation system and \LaTeX{} editor respectively, the user is done with \LaTeX{} system configuration}, no other manual intervention and configuration is required. If you have an old version of the \LaTeX{} compiler and want to install the new version, please \textbf{uninstall the old version first and then install the new version}.

\section{Questions Feedback}

For knowledgeable questions about \LaTeX{}, please check 
\href{https://github.com/mohuangrui/ucasthesis/wiki}{\LaTeX{} knowledge site} and 
\href{https://en.wikibooks.org/wiki/LaTeX}{\LaTeX{} Wikibook}.

For template compilation and styling issues, please \textbf{read this documentation carefully first, especially the "Frequently Asked Questions" (section ~\ref{sec:qa})}. If the problem still cannot be solved, please \textbf{understand and describe the problem first, then give feedback} to \href{https://github.com/sci-m-wang/NEU-Thesis-en/issues}{Github/NEU-Thesis-en/issues}.

We welcome your effective feedback on the shortcomings of the template, and we will keep improving it together. We hope that you will actively promote \LaTeX{} to your colleagues and do research more efficiently together.

\section{Template Download}

\begin{center}
    \href{https://github.com/sci-m-wang/NEU-Thesis-en}{Github/NEU-Thesis-en}: \url{https://github.com/sci-m-wang/NEU-Thesis-en}
\end{center}

